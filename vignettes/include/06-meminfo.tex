\section{Hardware Memory Information}

As of \thispackage version 2.0, some basic hardware memory information is 
accessible from several new \code{Sys.*()} functions:

\begin{center}
\vspace{0.2cm}
\begin{tabular}{ll} \hline\hline
Function & Description of Measurement \\ \hline
\code{Sys.meminfo()} & Total and free RAM (also buffers and cache if 
available)\\
\code{Sys.swapinfo()} & Total and free swap/page space.\\
\code{Sys.procmem()} & Total and peak RAM usage by current \R process.\\
\code{Sys.cachesize()} & Cache sizes \\
\code{Sys.cachelinesize()} & Cache line size \\
\hline\hline
\end{tabular}
\vspace{0.2cm}
\end{center}

Not all platforms are supported for any given function, and even among those that are, not all 
measurements are available across all platforms.  The platform with the best 
support in the \pkg{memuse} package is Linux.  This 
is largely because getting this information in Linux is very simple --- on top 
of the data itself being very well-defined --- compared to other platforms.  So 
get your shit together, other OS devs.

All of this data can be accessed by a standalone \Clang library located in 
\code{memuse/src/meminfo}.  
You can test the library by going to this directory and running \code{make}; also see the 
\code{README}.

All platforms get these values differently, and most of them are calculated in slightly 
different ways as well.  The remainder of this section will outline these differences, as a 
kind of angrily written reference standard, dedicated to the 
sisyphean tragedy of multiplatform software development.




\subsection{Terminology}

For clarity, we define some terms here.

RAM is ``fast'' storage on your computer (if you pretend that cache doesn't 
exist, anyway).  \emph{Free ram} is the portion of ram not being used, either 
by the OS (when this information is possible to get) or by programs, hackers, 
etc.

Swap is pretend-ram that resides on disk that you shouldn't ever actually use for anything.  
You're much better off having a program crash than sending you into a 40 hour swap death, where 
the computer becomes completely inaccessible..  How swap is used and/or created itself is 
platform dependent.  Swap is kind of the same thing as paging on 
Windows, and so \code{Sys.swapinfo()} is aliased as \code{Sys.pageinfo()}.  The 
two functions are identical within a given platform.  All operating systems basically treat 
swap/page files the same.  There are minor differences in allocation; for example, Mac OS X 
bizarrely uses unused space on the \code{/boot} partition for swap, for instance.

Cache is very fast temporary storage that data gets stuffed into along its way to being operated on 
by the processor.  For more information about cache, see the \pkg{pbdPAPI} package's vignette.  All 
operating systems basically do a lookup in some table somewhere after identifying your processor to 
find the cache sizes of your processor.  The same is true of cache line size, which is the size of 
chunks of memory handled by cache lookups.



\subsection{Sys.meminfo() and Sys.swapinfo()}

These functions return the amount of available and free ram for the former, and the amount of 
available and free swap for the latter.  How this happens is a labyrinthine mess.  Join me, won't 
you?


\subsubsection{Linux and FreeBSD}
In Linux and FreeBSD, the total available ram can be broken down into
\begin{itemize}
  \item truly unused (freeram)
  \item used for buffer cache (bufferram)
  \item used for file caching (cachedram)
  \item used by programs
\end{itemize}

By default, memuse will collapse the first three into \code{freeram}, because that's how most people 
think about it; and if you open up some kind of system monitor program, they probably do the 
exact same thing.  By \emph{truly unused}, we mean that nothing on the system has made any claim to 
it (according to the kernel).  The memory allocated for buffers and file cache is, well, file 
cache.  It's the stuff that makes loading a program the second time much faster than it is to load 
it the first time.  For more details, see \url{https://www.redhat.com/advice/tips/meminfo.html}.

These values should agree with (and in some cases, are taken directly from) those found in 
\code{/proc/meminfo}, with a small sample of the file provided:

\begin{center}
\begin{minipage}{.4\textwidth}
\begin{Output}
MemTotal:        8053556 kB
MemFree:         1761144 kB
Buffers:          808096 kB
Cached:          2327648 kB
SwapCached:            8 kB
\end{Output}
\end{minipage}
\end{center}

The actual implementation of \code{meminfo()} uses \code{sysinfo} (see \code{man 2 sysinfo}) for everything but cached memory, which is read directly from \code{/proc/meminfo}.


% \subsubsection{Windows}
% 
% Windows has a surprisingly good set of utilities.
% 
% 
% \subsubsection{Mac OS X}
% 



\subsubsection{Miscellaneous *NIX Variants}

Everybody seems to have their own little wiggle to \code{sysctl} for getting 
various memory statistics, and I just don't have the kind of time or resources 
to set up 100 different BSD-variant vm's to test that the man pages aren't lying 
to me.  As such, all *NIX's outside of Linux, FreeBSD, and Mac get only total ram and 
``free ram'', as reported by \code{sysconf}.

\code{Sys.swapinfo()} is not supported on these platforms at this time.




\subsection{Sys.procmem()}

We also provide a utility for discovering the amount of ram being used by the 
current \R process.  To use it, simply call:
\begin{lstlisting}[language=rr]
Sys.procmem()
\end{lstlisting}

This is utility is only supported on Linux, Windows, and Mac.  On Linux and 
Windows, the utility will also return the maximum amount of ram used by the 
current \R process.  You can see how this works by calling:

\begin{lstlisting}[language=rr]
library(memuse)

Sys.procmem()
x <- rnorm(1e7)
object.size(x)
Sys.procmem()
rm(x)
gc(FALSE)
Sys.procmem()
\end{lstlisting}

Just as an example, on my machine, this produces:
\vspace{-.6cm}
\begin{Output}
Sys.procmem()
## $size
## 134.488 MiB
## 
## $peak
## 134.488 MiB

x <- rnorm(1e7)

object.size(x)
## 76.294 MiB

Sys.procmem()
## $size
## 210.785 MiB
## 
## $peak
## 210.785 MiB

rm(x)
gc(FALSE)
##          used (Mb) gc trigger (Mb) max used (Mb)
## Ncells 324437 17.4     597831 32.0   407500 21.8
## Vcells 578176  4.5    9599428 73.3 10582203 80.8

Sys.procmem()
## $size
## 134.488 MiB
## 
## $peak
## 210.785 MiB
\end{Output}
